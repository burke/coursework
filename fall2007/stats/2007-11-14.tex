\documentclass[12pt]{amsart}

\begin{document}

\section*{Chapter 4 Summary}

\begin{enumerate}

\item Statistical Inference, Point Estimation

\item Hypothesis testing -- Statistical hypotheses, Type I and II errors. general procedure

\item Inference on the mean, variance known. Z-test, P-value approach, confidence interval estimation

\item Inference on the mean, variance unknown. T-test, confidence interval

\item Inference on the variance -- $X^{2}$ test

\end{enumerate}

\section*{Chapter 5 Outline}
\begin{enumerate}

\item Inference on the means, variance known, Z-test, Confidence Intervals

\item some more...

\end{enumerate}

\subsection*{Inference on the means}

\textbf{Example:} A product developer is interested in reducing the drying time of a primer paint. Two formulations of the paint are tested; formulation 1 is the standard chemistry, and formulation 2 has a new drying time. From experience, it is known that the s.d. of drying time is 8 minutes, and it is unaffected vy the addition of the new ingredient.

20 specimens are painted in random order: 10 with formulation 1, and another 10 with formulation 2. The two sample mean drying times are $\overline{x}_1 = 121$ and $\overline{x}_2 = 112$ minutes. Is the new ingredient effective (use $\alpha = 0.05$)?


\end{document}