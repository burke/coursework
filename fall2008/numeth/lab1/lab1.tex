\documentclass[10pt,letterpaper]{article}
\usepackage{geometry}
\usepackage{fullpage}
\usepackage{amsmath}
\begin{document}

\section*{\center MATH 2120 Lab 1\\ \small Burke Libbey (6840752)}

\begin{enumerate}

  \item %1

    \begin{enumerate}

      \item
        $E_{abs} = |3.141592 - 3.14| = 0.001592$ \\
        $E_{rel} = |3.141592 - 3.14|/|3.141592| = 0.00050674944$
      \item %1ii
        $E_{abs} = |1,000,000 - 999,996| = 4$ \\
        $E_{rel} = |1,000,000 - 999,996|/|1,000,000| = 0.000004$ 
      \item %1iii
        $E_{abs} = |0.000012 - 0.000009| = 0.000003$ \\
        $E_{rel} = |0.000012 - 0.000009|/|0.000012| = 0.25$ 
    \end{enumerate}

  \item %2
    \begin{enumerate}
      \item %2i
        $f(2.73) = 20.3 - 5(7.45) + 16.3 + 0.55 = -0.05$ \\
        $\delta = |0.011917 - (-0.05)|/|0.011917| * 100 = 519.5\%$
      \item %2ii
        $f(2.73) = 0.0320$ \\
        $\delta = |0.011917 - 0.0320|/|0.011917| * 100 = 168.5\%$
      \item The error in (b) is significantly lower than in (a). This is expected, since we've avoided subtracting similar small numbers.
    \end{enumerate}

  \item %3
    Done in double precision, this calculation results in $0$. \\
    \noindent A better representation would be:
    \[ \frac{1}{x(x+1)} \]
   
  \item %4
      $x_1 = x_0 - \frac{f(x_0)}{f'(x_0)} = 1.30769$ \\
      $x_2 = x_1 - \frac{f(x_1)}{f'(x_1)} =  $
      
\newpage

  \item %5
    \begin{enumerate}
      \item
        This sort of depends on your definition of `near'. It converges to a value on the same order of magnitude (8.4667).

        \begin{table}[htb] 
          \centering
          \caption{5a}
          \begin{tabular}{l l}
            \hline\hline
            n & $x_n$ \\
            \hline
            0 & 2 \\
            1 & 6 \\
            2 & 8.66667 \\
            3 & 8.84024 \\
            4 & 8.4645 \\
            5 & 8.4666 \\
            6 & 8.4667 \\
            7 & 8.4667 \\
            \hline
          \end{tabular} 
        \end{table}

        \item
          This converges to 1.2440, but it takes many more iterations than last time.

        \begin{table}[htb] 
          \centering
          \caption{5b}
          \begin{tabular}{l l l l l l l l}
            \hline\hline
            n & $x_n$    & n & $x_n$ & n & $x_n$\\
            \hline
            0 & 2 &      20& 1.2300 & 41& 1.2442 \\
            1 & 0.8571 & 21& 1.2556 & 42& 1.2434 \\
            2 & 1.7193 & 22& 1.2341 & 43& 1.2441 \\
            3 & 0.9586 & 23& 1.2521 & 44& 1.2436 \\
            4 & 1.5567 & 24& 1.2369 & 45& 1.2440 \\
            5 & 1.0357 & 25& 1.2497 & 46& 1.2437 \\
            6 & 1.4548 & 26& 1.2390 & 47& 1.2440 \\
            7 & 1.0932 & 27& 1.2480 & 48& 1.2437 \\
            8 & 1.3883 & 28& 1.2404 & 49& 1.2440 \\
            9 & 1.1356 & 29& 1.2468 & 50& 1.2440 \\
            10& 1.3437 & 30& 1.2414 & &\\
            11& 1.1665 & 31& 1.2459 & &\\
            12& 1.3133 & 32& 1.2421 & &\\
            13& 1.1887 & 33& 1.2453 & &\\
            14& 1.2923 & 34& 1.2426 & &\\
            14& 1.2047 & 35& 1.2449 & &\\
            15& 1.2778 & 36& 1.2430 & &\\
            16& 1.2161 & 37& 1.2446 & &\\
            17& 1.2677 & 38& 1.2433 & &\\
            18& 1.2242 & 39& 1.2444 & &\\
            19& 1.2606 & 40& 1.2434 & &\\

            
            \hline
          \end{tabular} 
        \end{table}

\newpage
        \item
          This one is back to a reasonable number of iterations, and converges to 1.2439.

        \begin{table}[htb] 
          \centering
          \caption{5c}
          \begin{tabular}{l l}
            \hline\hline
            n & $x_n$ \\
            \hline
            0 & 2 \\
            1 & 1.3093 \\
            2 & 1.2491 \\
            3 & 1.2443 \\
            4 & 1.2439 \\
            5 & 1.2439 \\
            \hline
          \end{tabular} 
        \end{table}


    \end{enumerate}
  \item %6
 
  \item %7

\end{enumerate}





\end{document}



%%% Local Variables: 
%%% mode: latex
%%% TeX-master: t
%%% End: 
